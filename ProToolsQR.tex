%!TEX TS-program = xelatex
%!TEX encoding = UTF-8 Unicode

%% Reference Card for ProTools

%% Author: Kevin Zembower

\documentclass[10pt]{article} 
\usepackage[portrait]{geometry}
%% \geometry{
%%   letterpaper,
%%   top = 0mm,
%%   width = 3in,                  %3x5 index card, portrait orientation
%%   height = 5in,
%%   %% left = 2.2125in,              %
%% }
\geometry{
  letterpaper,
  landscape,
  total={10in, 7.5in}
}

\usepackage{multicol}

%% \usepackage[utf8]{inputenc}     %For Unicode characters

\usepackage{fontspec}
\setmainfont{FreeSerif}         %Contains all these characters
%% \setmainfont{DejaVu Serif}      %Contains all these characters

%% \tracinglostchars=2 % Print a warning if a character is missing.

%% \DeclareUnicodeCharacter{⌘}{\keycmd}
%% \DeclareRobustCommand\keycmd{\keycmd}

%% \usepackage{showframe}

\pagestyle{empty}

%% \def\scaledmag#1{ scaled \magstep #1}
%% \font\titlefont=cmbx10 \scaledmag 1

\chardef\\=`\\
\chardef\{=`\{
\chardef\}=`\}
\chardef\]=`\]
\chardef\[=`\[

\def\cmd{⌘\enspace}             %U+2318
\def\shift{⇧\enspace}           %U+21E7
\def\opt{⌥\enspace}             %U+2325
\def\tab{⇥\enspace}             %U+21E5
\def\return{↩\enspace}         %U+21A9
\def\arrowkeyup{↑\enspace}      %U+2191
\def\arrowkeydown{↓\enspace}    %U+2193
\def\ctrl{⌃\enspace}            %U+2303

\def\versionpt{2024.6.0}             %ProTools version

% title - page title.  Argument is title text.
\outer\def\title#1{{\centerline{\textbf{#1}}}\vskip 1ex plus .5ex}

% section - new major section.  Argument is section name.
\outer\def\section#1{\par\filbreak
  \vskip 2ex plus 2ex minus 2ex {\textbf{#1}}\mark{#1}%
  \vskip 0ex plus 1ex minus 1.5ex}
  %% \vskip 2ex plus 1ex minus 1.5ex}

\newdimen\keyindent
% beginindentedkeys...endindentedkeys - key definitions will be
% indented, but running text, typically used as headings to group
% definitions, will not.
\def\beginindentedkeys{\keyindent=1em}
\def\endindentedkeys{\keyindent=0em}
\endindentedkeys

% kbd - argument is characters typed literally.  Like the Texinfo command.
\def\kbd#1{{#1}\null}	%\null so not an abbrev even if period follows

% key - definition of a key.
% \key{description of key}{key-name}
% prints the description left-justified, and the key-name in a \kbd
% form near the right margin.
\def\key#1#2{\leavevmode\hbox to \hsize{\vtop
  {\hsize=.65\hsize\rightskip=1em
  \hskip\keyindent\relax#1}#2\hfil}}

%% \def\key#1#2{\leavevmode\hbox to \hsize{\vtop  % Working version
%%   {\hsize=.75\hsize\rightskip=1em
%%   \hskip\keyindent\relax#1}#2\hfil}}

% threecol - like "key" but with two key names.
% for example, one for doing the action backward, and one for forward.
\def\threecol#1#2#3{\hskip\keyindent\relax#1\hfil&\kbd{#2}\hfil\quad
  &\kbd{#3}\hfil\quad\cr}

\newbox\metaxbox
\setbox\metaxbox\hbox{\kbd{M-x }}
\newdimen\metaxwidth
\metaxwidth=\wd\metaxbox

% metax - definition of a M-x command.
% \metax{description of command}{M-x command-name}
% Tries to justify the beginning of the command name at the same place
% as \key starts the key name.  (The "M-x " sticks out to the left.)
\def\metax#1#2{\leavevmode\hbox to \hsize{\hbox to .75\hsize
  {\hskip\keyindent\relax#1\hfil}%
  \hskip -\metaxwidth minus 1fil
  \kbd{#2}\hfil}}

\parindent 0pt
\parskip 1ex plus .5ex minus .5ex

\def\copyrightnotice{
\vskip 1ex plus 2 fill\begingroup\small
\centerline{Copyright \copyright\ \year\ E. Kevin Zembower}

This work is licensed under Creative Commons Attribution-ShareAlike
4.0 International.

Access the source and output of this document at
\texttt{https://github.com/kzembower/ProToolsQR}.
\endgroup}

\begin{document}
%% \footnotesize                   %Too big for 3x5 on page 2
%% \scriptsize
%% \tiny

\begin{multicols}{3}
  \title{ProTools Quick Reference Card}

  \centerline{(for version \versionpt)}

  \section{Key Binding Notation}

  \begin{tabular}{l@{\thinspace}ll@{\thinspace}l}
    \cmd          & Command         & \shift        & Shift           \\
    \tab          & Tab             & \opt          & Option          \\
    \ctrl         & Control         & \return       & Return or Enter \\
    \arrowkeyup   & Up arrow        & \arrowkeydown & Down arrow
  \end{tabular}

  \section{Session}
  \key{New}{\cmd N}
  \key{Save}{\cmd S}
  \key{Save as}{\ctrl\cmd S}
  
  \section{Zooming}
  \key{In/Out horiz}{T / R}
  \key{In/Out vert}{\cmd \opt , (comma)}
  \key{Selection}{\cmd \ctrl }
  \metax{Recall Zoom memory 1/2/3/4/5}{1/2/3/4/5}
  \key{All clips}{\opt A}
  \key{Fill window-selection}{\opt F}
  \key{Zoom toggle}{\ctrl E}

  \section{View}
  \key{Narrow mix}{\opt\cmd N}
  \key{Track marker lane}{\shift U}

  \section{Playing}
  \key{Start and Stop}{(spacebar)}
  \key{Selection}{\opt \[}
  \key{\dots to edit start}{6}
  \key{\dots from edit start}{7}
  \key{\dots to edit end}{8}
  \key{\dots from edit end}{9}
  \key{Solo cursor track}{\shift S}
  \key{Mute}{\cmd M}
  \key{Mute cursor track}{\shift M}
  \key{Selection}{\[}
  \key{Timeline selection}{\]}
  \key{Loop Enable}{\shift \opt L}
  \key{Toggle Looping}{\cmd \shift L}
  \key{Loop clip}{\opt \cmd L}
  
  \section{Bounce}
  \key{Session}{\opt\cmd B}
  \key{Track}{\opt\shift\cmd B}

  \section{Recording}
  \metax{Start}{\cmd-spacebar}
  \key{Stop}{spacebar}
  \key{Stop and discard}{\cmd . (period)}
  \key{Record-enable cursor track}{\shift R}

  \section{Tracks}
  \key{New}{\shift\cmd N}
  \key{Import audio}{\shift\cmd I}
  \key{Group}{\cmd G}
  \key{Duplicate}{\opt\shift D}
  \key{Move to new folder}{\opt\shift\cmd D}

  \section{Clips}
  \key{Edit Lock/Unlock}{\cmd L}
  \key{Time Lock/Unlock}{\ctrl \opt L}
  \key{Send to back}{\opt \shift B}
  \key{Bring to front}{\opt \shift F}
  \key{Group}{\opt \cmd G}
  \key{Ungroup}{\opt\cmd U}
  \key{Regroup}{\opt\cmd R}
  \key{Capture}{\cmd R}
  \key{Rename}{\opt\shift\cmd R}

  \section{Move Clip}
  \key{Start to cursor}{H}
  \key{End to cursor}{K}
  \key{Sync part by cursor}{J}
  \key{Nudge earlier/later}{, (comma)/. (period)}
  \key{Big nudge earlier/later}{M / / (slash)}

  \section{Trim}
  \key{Cursor to start/end}{A / S}
  \key{Break at cursor (Separate)}{B}
  \key{Heal separation}{\cmd H}
  \key{Strip silence}{\cmd U}

  \section{Fade}
  \key{Start to cursor}{D}
  \key{Cursor to end}{G}
  \key{Crossfade}{F}

  \section{Edit}
  \key{Cut}{\cmd X}
  \key{Copy}{\cmd C}
  \key{Paste}{\cmd V}
  \key{Undo}{\cmd Z}
  \key{Clear}{\cmd B}
  \key{Select all}{\cmd A}
  \key{Insert silence}{\shift\cmd E}
  
  \section{Modes}
  \key{Shuffle*}{F1 or \opt 1}
  \key{Slip*}{F2 or \opt 2}
  \key{Spot*}{F3 or \opt 3}
  \key{Grid}{F4 or \opt 4}
  \key{Cycle Edit modes}{' (apostrophe) }

  * can enable with Grid mode by pressing key and \kbd{F4} at the same
  time.

  \section{Cursor Movement}
  \key{Next clip boundary}{\tab}
  \key{Previous clip boundary}{\opt \tab}
  \key{Select next clip}{\ctrl \tab}
  \key{Select previous clip}{\ctrl \opt \tab}
  \key{Session start}{\return}
  \key{Session end}{\opt \return}
  \key{Beginning/End of selection}{\arrowkeyup~/~\arrowkeydown}
  \key{Goto song start}{\ctrl \return}

  \section{Tools}
  \key{Zoomer}{F5 or \cmd 1}
  \key{Trimmer}{F6 or \cmd 2}
  \key{Selector}{F7 or \cmd 3}
  \key{Grabber}{F8 or \cmd 4}
  \key{Scrubber}{F9 or \cmd 5}
  \key{Pencil}{F10 or \cmd 6}
  \key{Smart}{F6 \& F7}
  \key{Cycle Edit tools}{Esc}

  \section{Extend Selection}
  \key{To start of session}{\shift \return}
  \key{To end of session}{\shift \opt \return}

  \section{Link Edit Selection}
  \key{\dots to Timeline}{\shift /}
  \key{\dots to Track}{\shift T}

  \section{Beats}
  \key{Beat Detective}{\cmd 8}
  \key{Identify Beats}{\cmd I}

  \copyrightnotice
  
\end{multicols}

\end{document}
